%====================
% EXPERIENCE A
%====================
\subsection{{Data Scientist \hfill 08-2022 $-$ Present}}
\subtext{King \hfill London - United Kingdom}
\begin{zitemize}
\item 
I have work in two different game teams: business performance unit of ``Bubble Witch 3 Saga'' (BW3S) and Product Analytics of ``Candy Crush Saga'' (CCS). In both games I wrote and presented several reports for the stakeholders assessing metrics such as daily active users, gross revenue, ads revenue, daily paying users and game-rounds played by players. 
I have explored thoroughly very large datasets (big data) using SQL (in GCS Big Query) and Python in order to track app events and evaluate completion rates and players engagement in game features and events
I have wrote Python code for data products in analytics with multiple purposes (data visualization, ETL, feature simulation, forecasting, etc).
I've also helped in the design of A/B test experiments and custom tracking events of client apps to evaluate new game features developed by game devs in BW3S. 
\end{zitemize}

%====================
% EXPERIENCE B
%====================
\subsection{{Data Scientist / Marketing Analytics \hfill 08-2021 $-$ 07-2022}}
\subtext{Magazine Luiza \hfill São Paulo, SP - Brazil}
\begin{zitemize}
\item  In the Advanced Analytics team we used data of our marketplace for planning and optimization of several operations such
as logistics, supply chain, marketing, recommendation systems and so on. I coded in Python, using PySpark for big data processing, and used SQL with BigQuery on a daily basis. We had constant communication with the data engineering team to optimize deployment and data processing of our products. Our codes were versioned in Gitlab. I presented insights and reports weekly to directors, managers and operational stakeholders in order to update the development of our team tasks.
\item I have programmed automatic maintenance and intraday report insights of smart shopping campaigns with Google Ads API. The data provides business insights to our Marketing team and reliable real-time information for marketing metrics evaluation.
\item I have deployed rule-based models to optimize shopping campaign performance. My code evaluates a large amount of products considering historic adcost and ROI performance, and identifies low performance products that drags cash margin down. The identified products are automatically removed from paid shopping ads during a limited amount of time.
\item I have analysed sellers performance on the marketplace in order to calculate their regional conversion rates. I have segmented sellers according to their states and identified a direct relation between conversion rate and distance-to-client. Using this information, we have limited ad impressions of small sellers to be shown only in a certain local radius, aiming to optimize investment and increase overall conversion rate and local delivery rates.
\end{zitemize}
%====================
% EXPERIENCE C
%====================
\subsection{{Data Science Instructor \hfill 03-2022 $-$ 06-2022}}
\subtext{Ultima School \hfill São Paulo, SP - Brazil}
\begin{zitemize}
\item I have developed instructional material (videos and tutorials) for basic Python programming, Python development interfaces (such as Jupyter Notebooks, VS Code and Google Colab) and basic statistics for data science beginners.
\item I recorded live webinars with students using a code-along approach to demonstrate all the basic tools of python programming and basic concepts of statistics.
\item I have written the Data Processing module material with 4 chapters focusing on all the basic tools of the pandas library. My material contains practical examples and exercises using an open dataset of Brazilian E-commerce from Olist, available in Kaggle. I also recorded webinars with the students solving exercises and showing how to use pandas operations and table manipulation.
\end{zitemize}

%====================
% EXPERIENCE D
%====================
\subsection{{Assitant Professor \hfill 03-2017 $-$ 08-2021}}
\subtext{Instituto Federal Sul-rio-grandense (IFSul) \hfill Sapiranga, RS - Brazil}
\begin{zitemize}
\item I taught several courses for technicians on Electromechanics, Electro-electronics and Electrotechnics. Some of the educational content produced by me can be found in the institutional YouTube page. 
\item Besides teaching activities, I have advised students on some research projects: an USB charger powered by a photovoltaic panel, using a DC/DC buck converter; a machine learning algorithm to identify Twitter users with depression symptoms; a face detector that identifies the use of masks in the context of COVID-19.
\item Other activities include coordination of a music group, production of video-training on G-Suite tools and collaboration on internal events and committees.
\end{zitemize}

%====================
% EXPERIENCE E
%====================
%\subsection{{ROLE / PROJECT E \hfill MMM YYYY --- MMM YYYY}}
%\subtext{company E \hfill somewhere, state}
%\begin{zitemize}
%\item In lobortis libero consectetur eros vehicula, vel pellentesque quam fringilla.
%\item Ut malesuada purus at mi placerat dapibus.
%\item Suspendisse finibus massa eu nisi dictum, a imperdiet tellus convallis.
%\item Nam feugiat erat vestibulum lacus feugiat, efficitur gravida nunc imperdiet.
%\item Morbi porta lacus vitae augue luctus, a rhoncus est sagittis.
%\end{zitemize}
